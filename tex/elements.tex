\documentclass[table]{scrartcl}
\usepackage[utf8]{inputenc}
\usepackage[english]{babel}
\usepackage{acro}
\usepackage{articledh}
\usepackage{courier}
\usepackage{minted}
\usepackage{tcolorbox}
\usepackage{xcolor}
\usepackage{colortbl}
\usepackage[]{hyperref}

\hypersetup{
  pdftitle={Elements - Documentation},
  pdfauthor={Datahouse AG},
  pdfcreator={Datahouse AG},
  hidelinks=true
}

\person{Markus Wanner\\markus.wanner@datahouse.ch}

\acsetup{first-style=short}
\DeclareAcronym{ajax}{
  short = AJAX,
  long  = Asynchronous JavaScript and XML,
  class = abbrev
}

\DeclareAcronym{api}{
  short = API,
  long  = Application Programming Interface,
  class = abbrev
}

\DeclareAcronym{cms} {
  short = CMS,
  long = Content Management System,
  class = abbrev
}

\DeclareAcronym{http}{
  short = HTTP,
  long  = Hypertext Transfer Protocol,
  class = abbrev
}

\DeclareAcronym{ide} {
  short = IDE,
  long = Integrated Development Environment,
  class = abbrev
}

\DeclareAcronym{pac} {
  short = PAC,
  long = Presentation Abstraction Control,
  class = abbrev
}

\DeclareAcronym{php} {
  short = PHP,
  long = PHP,
  class = abbrev
}

\DeclareAcronym{json}{
  short = JSON,
  long  = JavaScript Object Notation,
  class = abbrev
}

\DeclareAcronym{rest}{
  short = REST,
  long  = Representational State Transfer,
  class = abbrev
}

\DeclareAcronym{yaml}{
  short = YAML,
  long  = YAML Ain't Markup Language,
  class = abbrev
}

\DeclareAcronym{mvc}{
  short = MVC,
  long  = Model View Controller,
  class = abbrev
}

\newcommand{\code}[1]{\texttt{#1}}

% styling for the json examples
\tcbuselibrary{minted,skins}
\newtcblisting{jsonexample}{
  listing engine=minted,
  colback=gray!10,
  boxrule=0pt,
  listing only,
  minted style=friendly,
  minted language=javascript,
  minted options={linenos=true,texcl=true},
  left=7mm
}

\begin{document}

\doctitle{Elements - Documentation}

\section{Introduction}
\subsection{Goals and Design Decisions}
Elements is intended to be a building block for a highly customizable
Content Management System (\ac{cms}), available in the form of a
\ac{php} library ready to be embedded in a web application.

It is designed to be database agnostic and has very little
requirements on the underlying data storage. By design, a key-value
store or even a file-system should be sufficient. Features offered by
(and usually implemented in) the database (like triggers, constraints
and indices) therefore had to be implemented in Elements, where
needed.

User content is stored in elements, which in turn are organized in a
tree. Every user-visible page is an element, but then there are other
types of elements like snippets and collections. Part of the tree
usually mirrors the sitemap or page structure of the website to
manage. Other parts of the tree may contain additional content
(usually snippets) which may be referenced by a page or other
syelements. Multiple pages may point to a single snippet, so this allows
repetitive content to be managed at a single place. Examples are
contact persons, header or footer elements or a list of news needs to
be displayed in different forms on multiple pages.

Another early design decision was to use Froala for comfortable
in-line content editing in the browser. This is a third-party
component written in JavaScript running directly on the visitor's
browser. This in turn also led to the decision to use an implicit
(background) auto-save mechanism.

\subsection{Current State}
Not all of the original ideas and concepts have been implemented, and
this shows in the projects using Elements. This document intends to
describe the concepts and ideas behind. Whenever an idea or concept is
implemented only partially or not at all, this should be mentioned.

\section{Overview}
\subsection{Requirements}
Elements mainly consists of a \ac{php} library that can be included
via composer. It also offers a Dockerfile that's currently copied via
a strange ant helper target that must be included from an it-build
project. See existing applications WWP and WWW for how to setup ant.

At least version 7 of PHP is required. The docker image takes care of
that, but you might need to adjust your ac{ide}.

Froala is embedded in Elements and ships with it.

\section{Building Blocks}
This section describes the building blocks of Elements. They are
grouped according to the \ac{pac} concept, which is somewhat similar
to \ac{mvc}, but cleaner in that the controller clearly is in between
the presentation and abstraction layer. The latter two must not
directly communicate with each other.

\subsection{Presentation}

\subsubsection{Twig Templates}
Elements uses Twig to combine data from the controller with actual
markup. Twig in turn defines a few commands that extend HTML, but the
Twig template is very close to the effectively rendered document.

\subsubsection{Page Definitions}
The page definition in a \ac{php} class representing the actual
template the user (or to be more specific: the Elements application
admin) can choose from. It links the HTML-like Twig template with the
\ac{php} world and provides additional information for Elements, like
what are editable fields, what other (types of) elements can be
referenced, and defines the sub-elements that can possibly be added.

\subsection{Abstraction}

\subsubsection{Storage Adapter}
Elements is designed to work with multiple storage adapters, which
provide a common interface to whatever storage engine is used
underneath. They all implement the common \code{IStorageAdapter}
interface. Only one storage adapter can be used at a time.

The choice is pretty simple at the moment, as only the
\code{YamlAdapter} is fully implemented. This is a plain file-system
based storage adapter using \ac{yaml} to structure its data in files.

An \code{SqliteAdapter} is partially implemented as well and is used
in certain unit tests. However, it's not complete and no deployed
application ever used it up until now.

\subsubsection{Elements Cache}
In between the storage adapter and the controllers sits a cache layer
for improved performance. This should be considered part of the
abstraction. Elements takes care of invalidating entries in this cache
in case of modifications via Elements itself If an external component
or script changes the underlying storage, this cache needs to be
invalidated manually or via a corresponding API call (see method
\code{recreateCacheData} of the storage adapter). When logged in as an
admin, the URL /admin/cache provides some statistics and offers the
option to invalidate this cache.

\subsubsection{Storage Migrator}
Actually a somewhat separate tool, invoked only via the admin
front-end, this component takes care of migrating existing data to a
new format, should the storage format of Elements itself
change. Please visit /admin/setup to perform storage migrations, if
necessary.

\subsection{Control}

\subsection{Authentication Handler}
The authentication handler takes care of checking passwords, loading
session data and logging in or out a user. It is extensible by the
application and basically returns a User object for Elements to work
with.

This user is passed on to the Authorization Handler for permission
checks and may be referenced from elements to provide a change
history.

\subsection{Authorization Handler}
Given a User, an Element and a right (the right to view or to modify,
for example), the authorization handler decides which versions of the
given Element are visible or may be modified.

This may well include multiple versions, either featuring content in
different languages, in different states (editing, published) or even
historic versions. The authorization handler is supposed to return all
versions the user is granted the given right to.

Ideally, this Authorization Handler is the only component in the
entire system that cares about access permissions. An application may
well extend this handler and introduce new rights and
permissions. Elements itself only defines two hard-coded rights:
``view'' and ``modify''.

For each page visit, the authorization handler will be called at least
once. If the page includes references to other elements, it will
additionally be invoked once per referenced element.

\subsection{Content Selector}
Given a set of authorized elements possibly including different
languages as well as older versions, the content selector determines
the best variant or version of the element to show. In \ac{http}
parlance, this is often termed ``content negotiation''. However, in
Elements, the Content Selector additionally selects the newest
version to display.

Again, this selector can be overridden by the application. In
practice, the default selector usually is sufficient.

\end{document}
